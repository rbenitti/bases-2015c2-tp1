\section{Introduccion}

El objetivo del trabajo es modelar, dise\~nar e implementar la base de datos \textbf{\emph{RUAT}}
(Registro Único de Accidentes de Tránsito).

La misma consiste en un Sistema Nacional de Registro de Accidentes Viales. Tal como su nombre lo indica, el mayor interés
está enfocado en la posibilidad de efectuar consultas respecto a los accidentes de tránsito acaecidos. Quiénes son los
conductores de los vehículos causantes de los accidentes, en qué lugares se produjeron, quiénes son los testigos,
número de patente de los vehículos, etc., son ejemplos de datos importantes
que se desea conocer y por los que se quiere buscar.

Además de llevar un registro de accidentes, se desea incorporar como información adicional los datos relacionados con
infracciones de tránsito y antecedentes penales de las personas involucradas en los siniestros.

Por último, se quiere contar también con información del parque automotor que circula por el país: compa\~nía de seguros,
categoría, antigüedad, cobertura de cada vehículo, etc.

\subsection{Motor de base de datos}

El motor de base de datos elegido fue SQL Server porque fue el utilizado por la cátedra durante el laboratorio de datos
de bases de datos relacionales. Además está instalado en las máquinas del laboratorio, lo cual facilita la demostración
presencial. De cualquier manera, consideramos que no hubiera sido muy diferente realizar otra elección, ya que los otros
motores: \textbf{\underline MySQL}, \textbf{\underline SQLite} se asemejan en cuanto a la sintaxis y funcionalidades a la
hora de implementar bases de datos estándar.