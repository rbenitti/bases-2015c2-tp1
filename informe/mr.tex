\section{Modelo Relacional}

\begin{itemize}

\item Lugar(\underline{idLugar}, nombre, tipoPavimento, longitud, velocidaMaxima, tipo)
  \begin{itemize}
  \item[] PK = CK = \{idLugar\}
  \end{itemize}


\item Calle(\underline{idLugar}, provincia, ciudad)
  \begin{itemize}
  \item[] PK = CK = FK =\{idLugar\}
  \end{itemize}


\item Ruta(\underline{idLugar}, esNacional)
  \begin{itemize}
  \item[] PK = CK = FK =\{idLugar\}
  \end{itemize}


\item Persona(\underline{dni}, nombre, apellido, nacionalidad, fechaNacimiento, domicilio, numeroLic, clases, vencimiento, otorgamiento)
  \begin{itemize}
  \item[] PK = \{dni\}
  \item[] CK = \{dni, numeroLic\}
  \end{itemize}


\item Vehiculo(\underline{patente}, anio, tipo, marca, modelo, categoria, titular, aseguradora, tipoCobertura, vigenciaSeguro)
\begin{itemize}
  \item[] PK = CK = \{patente\}
  \item[] FK = \{dniDuenio\}
  \end{itemize}


\item Modalidad(\underline{idModalidad}, descripcion)
  \begin{itemize}
  \item[] PK = CK = \{idModalidad\}
  \end{itemize}


\item Accidente(\underline{idAccidente}, fecha, idLugar, altura, fechaDenuncia, comisaria, oficial, idModalidad)
  \begin{itemize}
  \item[] PK = CK = \{idAccidente\}
  \item[] FK = \{idLugar, idModalidad\}
  \end{itemize}

\item Infraccion(\underline{idInfraccion}, fecha, idLugar, altura, tipo, oficial)
  \begin{itemize}
  \item[] PK = CK = \{idInfraccion\}
  \item[] FK = \{idLugar\}
  \end{itemize}

\item HabilitacionConduccion(\underline{dni} , patente)
  \begin{itemize}
  \item[] PK = CK = \{(dni, patente)\}
  \item[] FK = \{dni, patente\}
  \end{itemize}

\item Testigo(\underline{idAccidente}, dni)
  \begin{itemize}
  \item[] PK = CK = \{(idAccidente, dni)\}
  \item[] FK = \{idAccidente, dni\}
  \end{itemize}

\item InfraccionVehiculoPersona(\underline{idInfraccion}, \underline{dni}, patente)
  \begin{itemize}
  \item[] CK = \{(idInfraccion, dni), (idInfraccion, patente)\}
  \item[] PK = \{(idInfraccion, dni)\}
  \item[] FK = \{idInfraccion, dni, patente\}
  \end{itemize}

\item AccidenteVehiculoPersona(\underline{idAccidente}, \underline{dni}, patente, rolPersona)
  \begin{itemize}
  \item[] PK = \{(idAccidente, dni), (idAccidente, patente)\}
  \item[] PK = \{(idAccidente, dni)\}
  \item[] FK = \{idAccidente, dni, patente\}
  \end{itemize}


\item Peritaje(\underline{idPeritaje}, \underline{idAccidente}, perito, fecha, motivo, resultado)
  \begin{itemize}
  \item[] PK = CK = \{idPeritaje, idAccidente\}
  \item[] FK = \{idAccidente\}
  \end{itemize}

\end{itemize}