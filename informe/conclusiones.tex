\section{Conclusiones}

La base de datos obtenida satisface los requerimientos pautados por el enunciado. La implementación
es el resultado de un proceso que consta de 4 pasos:
\begin{itemize}
\item{Relevamiento de objetivos}
\item{Diagrama de entidad-relación}
\item{Modelo de entidad relación}
\item{Implementación}
\end{itemize}
Definir claramente los objetivos permitió abstraer el problema, transformándolo a un modelo de entidades y relaciones.
El DER es una herrmienta súmamente útil, ya que permite resumir la información de forma
visual y anticipar problemáticas que pueden llegar a surgir más adelante. La navegabilidad del diagrama determina
el alcance de las queries que quieran hacerse en el futuro.
El modelo de entidad relación es el paso intermedio entre el armado del diagrama y la implementación final.
Consideramos que la realización del trabajo práctico fue enriquecedor para nuestra formación, ya que nos permitió recorrer
por completo el proceso de creación de una base de datos relacional.
